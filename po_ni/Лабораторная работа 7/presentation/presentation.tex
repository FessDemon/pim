%\documentclass[aspectratio=169]{beamer}
\documentclass[aspectratio=169, handout]{beamer}
\usepackage[utf8]{inputenc}
\usepackage[T2A]{fontenc}
\usepackage[russian]{babel}
\usepackage{pgfplots}
\usepackage{minted}
\usepackage{graphicx}

% Тема презентации
\usetheme{Madrid}
\definecolor{myblue}{RGB}{0, 102, 204}
\setbeamercolor{structure}{fg=myblue}

% Настройка pgfplots
\pgfplotsset{compat=1.18}

\title{Логистическая регрессия}
\author{Автор: Дмитрий Бояринцев}
\date{\today}

\begin{document}
	
	% Титульный слайд
	\begin{frame}
		\titlepage
	\end{frame}
	
	% Введение
	\begin{frame}{Введение}
		\begin{block}{Что такое логистическая регрессия?}
			Логистическая регрессия — это статистический метод анализа данных, используемый для предсказания вероятности принадлежности объекта к одному из классов.
		\end{block}
		\pause
		\begin{itemize}
			\item Применяется для задач бинарной классификации.
			\pause
			\item Основана на логистической функции (сигмоиде).
			\pause
			\item Является частным случаем обобщенной линейной модели.
		\end{itemize}
	\end{frame}
	
	% Математическая модель
	\begin{frame}{Математическая модель}
		Логистическая регрессия моделирует вероятность $ P(y=1 | \mathbf{x}) $ как:
		\[
		P(y=1 | \mathbf{x}) = \frac{1}{1 + e^{-(\beta_0 + \beta_1 x_1 + \dots + \beta_n x_n)}}
		\]
		\pause
		Где:
		\begin{itemize}
			\item $ \beta_0, \beta_1, \dots, \beta_n $ — параметры модели.
			\item $ x_1, x_2, \dots, x_n $ — признаки объекта.
		\end{itemize}
	\end{frame}
	
	% График сигмоиды
	\begin{frame}{График сигмоиды}
		\begin{center}
			\begin{tikzpicture}
				\begin{axis}[
					xlabel={$x$},
					ylabel={$P(y=1)$},
					xmin=-6, xmax=6,
					ymin=0, ymax=1.1,
					domain=-6:6,
					samples=100,
					grid=both,
					width=0.8\textwidth,
					height=0.5\textwidth
					]
					\addplot[blue, thick] {1 / (1 + exp(-x))};
				\end{axis}
			\end{tikzpicture}
		\end{center}
	\end{frame}
	
	% График из CSV
	\begin{frame}{График на основе данных из CSV}
		\begin{center}
			\begin{tikzpicture}
				\begin{axis}[
					xlabel={Признак},
					ylabel={Целевая переменная},
					legend pos=north west,
					width=0.8\textwidth,
					height=0.5\textwidth
					]
					\addplot table[x=x, y=y, col sep=comma] {data.csv};
					\legend{Данные}
				\end{axis}
			\end{tikzpicture}
		\end{center}
	\end{frame}
	
	% Алгоритм обучения
	\begin{frame}[fragile]{Алгоритм обучения}
		\begin{block}{Шаги обучения логистической регрессии}
			\begin{enumerate}
				\item<1-> Инициализация параметров $ \beta $.
				\item<2-> Вычисление предсказаний с использованием сигмоиды.
				\item<3-> Обновление параметров с помощью градиентного спуска.
			\end{enumerate}
		\end{block}
	\end{frame}
	
	% Листинг кода
	\begin{frame}[fragile]{Пример кода на Python}
		\begin{minted}[linenos, fontsize=\small]{python}
			import numpy as np
			from sklearn.linear_model import LogisticRegression
			
			# Данные
			X = np.array([[1, 2], [2, 3], [3, 4]])
			y = np.array([0, 0, 1])
			
			# Обучение модели
			model = LogisticRegression()
			model.fit(X, y)
			
			# Предсказание
			print(model.predict([[4, 5]]))
		\end{minted}
	\end{frame}
	
	% Преимущества и недостатки
	\begin{frame}{Преимущества и недостатки}
		\begin{columns}
			\begin{column}{0.5\textwidth}
				\begin{block}{Преимущества}
					\begin{itemize}
						\item Простота реализации.
						\item Интерпретируемость результатов.
					\end{itemize}
				\end{block}
			\end{column}
			\begin{column}{0.5\textwidth}
				\begin{block}{Недостатки}
					\begin{itemize}
						\item Чувствительность к выбросам.
						\item Не подходит для многоклассовой классификации без модификаций.
					\end{itemize}
				\end{block}
			\end{column}
		\end{columns}
	\end{frame}
	
	% Заключение
	\begin{frame}{Заключение}
		Логистическая регрессия — это мощный инструмент для решения задач бинарной классификации. Она проста в реализации и интерпретации, но имеет ограничения, которые необходимо учитывать при выборе модели.
	\end{frame}
	
	% Спасибо за внимание
	\begin{frame}{Спасибо за внимание!}
		\begin{center}
			\Huge Спасибо за внимание!
		\end{center}
	\end{frame}
	
\end{document}